% sherman 1
% $Log: abstract.tex,v $
% Revision 1.1  93/05/14  14:56:25  starflt
% Initial revision
% 
% Revision 1.1  90/05/04  10:41:01  lwvanels
% Initial revision
% 
%
%% The text of your abstract and nothing else (other than comments) goes here.
%% It will be single-spaced and the rest of the text that is supposed to go on
%% the abstract page will be generated by the abstractpage environment.  This
%% file should be \input (not \include 'd) from cover.tex.

% TITLE: 
% Detecting Student Progress during Programming Activities by Analyzing Edit Operations on their Blocks-Based Programs

Investigation into students' programming processes has emerged as a nascent yet critical area to inform teaching methodology and learning models. Previous studies have used fine-grained instrumentation to delve into previously inaccessible student behaviors. These behavior data, observations of small changes in work over time, filled in the stories that led up to the final artifacts, and were not captured by looking at artifacts alone. 

Blocks languages afford new dimensions of interaction with code. In this work, student progress through an activity was investigated. Fine-grain data based on edit operations in blocks programming were collected, and analysis methods and visualizations for those data were designed and implemented. One class of behaviors investigated was ``flailing,'' which included patterns of repetitive or non-productive changes, disengagement, and potentially lack of problem understanding. One of the visualizations created was a prototype real-time tool for teachers, to help the the teacher to assess their students engagement, progress, and degree of flailing live during a lab session. This tool may be helpful to improve teaching effectiveness by aiding in classroom orchestration, empowering the teacher to better utilize their resources in the moment, to better keep all students on track.
