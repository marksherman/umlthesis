% sherman 1
% $Log: abstract.tex,v $
% Revision 1.1  93/05/14  14:56:25  starflt
% Initial revision
% 
% Revision 1.1  90/05/04  10:41:01  lwvanels
% Initial revision
% 
%
%% The text of your abstract and nothing else (other than comments) goes here.
%% It will be single-spaced and the rest of the text that is supposed to go on
%% the abstract page will be generated by the abstractpage environment.  This
%% file should be \input (not \include 'd) from cover.tex.

% TITLE: 
% Detecting Student Progress during Programming Activities by Analyzing Edit Operations on their Blocks-Based Programs

Investigation into students' programming processes has long been a critical area to inform teaching methodology and learning models. Recent technology has created a nascent field of using fine-grained instrumentation---observations of small changes in work over time---to delve into previously inaccessible student behaviors. These behavior data filled in stories that led up to the final program code artifacts; these stories could not be captured by looking at these artifacts alone. 

Also recently, blocks-based programming languages have become popular and practical for pedagogical programming environments. These environments afford new dimensions of learner interaction with code. 

In this work, student progress during a programming challenge activity was investigated. Fine-grained data were collected based on edit operations in students' blocks-programming work. Analysis methods and visualizations for those data were designed and implemented. A construct of a class of behavior called ``flailing'' was developed, and was operationalized to include patterns of repetitive or non-productive changes. This suggested disengagement and potential lack of problem understanding. A simple algorithm was developed to demonstrate that ``flailing'' can be automatically assessed.

In addition to visualizations of individual student progress, a prototype of a classroom ``dashboard'' was developed. This is intended to be used as real-time tool for teachers, to help teachers to assess their students engagement, progress, and degree of flailing during a live lab session. This tool may be helpful to improve teaching effectiveness by aiding in classroom orchestration, empowering the teacher to better employ their resources in the moment, to better keep all students on track.

From these data, three patterns of student were discovered: those who smoothly completed the assignment, those who worked their way through with periods of flailing and periods of success, and those who only flailed and never succeeded. This rating system allowed for the development of a moment-by-moment analysis of flailing, which may provide a real-time assessment of students engagement with the activity, and serve as a tool to further empower the teacher to orchestrate their resources during a lab session.

%old
%Investigation into students' programming processes has emerged as a nascent yet critical area to inform teaching methodology and learning models. Previous studies have used fine-grained instrumentation to delve into previously inaccessible student behaviors. These behavior data, observations of small changes in work over time, filled in the stories that led up to the final artifacts, and were not captured by looking at artifacts alone. 

%Blocks languages afford new dimensions of interaction with code. In this work, student progress through an activity was investigated. Fine-grain data based on edit operations in blocks programming were collected, and analysis methods and visualizations for those data were designed and implemented. One class of behaviors investigated was ``flailing,'' which included patterns of repetitive or non-productive changes, disengagement, and potentially lack of problem understanding. One of the visualizations created was a prototype real-time tool for teachers, to help the the teacher to assess their students engagement, progress, and degree of flailing live during a lab session. This tool may be helpful to improve teaching effectiveness by aiding in classroom orchestration, empowering the teacher to better utilize their resources in the moment, to better keep all students on track.
