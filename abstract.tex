% sherman 1
% $Log: abstract.tex,v $
% Revision 1.1  93/05/14  14:56:25  starflt
% Initial revision
% 
% Revision 1.1  90/05/04  10:41:01  lwvanels
% Initial revision
% 
%
%% The text of your abstract and nothing else (other than comments) goes here.
%% It will be single-spaced and the rest of the text that is supposed to go on
%% the abstract page will be generated by the abstractpage environment.  This
%% file should be \input (not \include 'd) from cover.tex.

\begin{quote}
Can we automatically detect when a student needs help?
\end{quote}
Investigation into students' programming processes has emerged as a nascent yet critical area to inform teaching methodology and learning models. Previous studies have used fine-grained instrumentation to delve into previously inaccessible student behaviors. These behavior data, observations of small changes in work over time, filled in the stories that led up to the final artifacts, and were not captured by looking at artifacts alone. 
% TODO "dive" vs "delve"

Blocks languages afford new dimensions of interaction with code, and this project applies fine-grained instrumentation to a visual, blocks-based programming language. This work uses learning analytics style of data mining to extract signals from the interaction data that can indicate certain student behaviors are occurring. The first behavior under investigation is flailing- where a student haphazardly edits code with no reason, hoping to stumble into a working solution. Real-time detection of flailing could be a useful tool for classroom instructors.
