% sherman 1
% $Log: abstract.tex,v $
% Revision 1.1  93/05/14  14:56:25  starflt
% Initial revision
% 
% Revision 1.1  90/05/04  10:41:01  lwvanels
% Initial revision
% 
%
%% The text of your abstract and nothing else (other than comments) goes here.
%% It will be single-spaced and the rest of the text that is supposed to go on
%% the abstract page will be generated by the abstractpage environment.  This
%% file should be \input (not \include 'd) from cover.tex.

Investigation into students' programming processes has emerged as a critical and nascent area to inform teaching methodology and learning models. Previous studies have used fine-grained instrumentation to dive into previously inaccessible student behaviors. These behavior data, observations of small changes in work over time, filled in the stories that led up to the final artifacts, and were not captured by looking at artifacts alone. This project applies this fine-grained instrumentation to a visual, blocks-based programming language. Blocks languages afford new dimensions of interaction with code. This work uses learning analytics style of data mining to extract signals from the interaction data that can indicate certain student behaviors are occurring. The first behavior under investigation is flailing- where a student haphazardly edits code with no reason, hoping to stumble into a working solution. Real-time detection of flailing could be a useful tool for classroom instructors.

A modified version of MIT App Inventor was developed to instrument user programming work. This version recorded students’ every change as they built apps. Feature extraction methods were developed to generate measures suitable for data mining.

Two activities were designed to elicit code modification and adaptation. The experiment was conducted with middle school students in classroom and summer camp environments, totaling 154 participants. Data analysis is yielding signals in these data indicating flailing conditions, as assessed by human experts. This could lead to a real-time “teacher dashboard,” to detect flailing students in real-time.
