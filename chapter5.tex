%\chapter{Discussion}
\chapter{Plan for Remaining Work} % Proposal chapter

Education research is often fraught with unreliabilty. At this point, all data have been collected, and all analysis tools have been built. Preliminary analysis has been conducted that demonstrated the efficacy of the data extraction and processing pipeline. This anwered my first class of research questions, concerning the creation of a snapshot system, and the efficacy of the fine-grain data it could provide. 

Moving forward, with a strong $n$, and a coherent theoretical framework, I will spend the next period creating a careful and complete analysis of flailing patterns within these data, and undergo the generative, creative task of building mathematical and programatic tools to generalize detection of this behavior over new data. 

Expansions to this document will include additional literature on learning anlytics, off-task behavior, and learning pathways, including: \cite{martin2013nanogenetic}, \cite{baker2004off} and an expanded literature assessment of \citet{berland-2013}. Additional literatue on programming languages in schools, specifically \cite{saez2016visual}. There will also be a survey of integrated teacher dashboard systems, to equip the this work to be developed into a real-time teacher tool. 

The most significant enhancement to this document will be the rigerous treatment of the analysis tools. These tools, at the time of this proposal, were mostly complete, but lacking the final unity that the remaining analysis tasks will bring. With that, relevant source code has been included, but the explanation thereof is intentionally reserved. 

This work is intended to be completed early in the spring semester 2017.
	
%\section{Conclusions}


%\section{Recommendations} 
\label{sec:recommendations}
We do not recommend using git as a database. Further recommendaitons of legitimate scientific and educational merit will be added as part of the full dissertation.


 \section{Future Work, beyond this dissertation} % TODO reduce to just "future work" after proposal
\label{sec:futurework}

% This section entails work that may be conducted beyond the dissertation, and will included, sans this sentence, in the final thesis document as well as here in the proposal.

% % Copied straight from chap 2 for the moment, but needs to be mentioned. good stuff.
As \citet{petre-2006} observed, experts and novices write and read secondary notation in different ways, and in App Inventor, poor planning of that secondary notation by a novice can result in increased viscosity. Increased viscosity, we hypothesize, may result in tinkering behavior, as it adds time and cognitive load to the user's task. Further investigation along this line is reserved for future work.

Better fine-grain collection of snapshot data. In this work, the entire project was captured on every change, which created a need to reverse-engineer what changes occured between snapshots post-facto. Upcoming updates to Blockly will allow for easier access to succinct event descriptions, potentially allowing only the changes themselves to be sent, effectively reducing the transmission payload and increasing transparency of collected data. Such improvements can make future collection experiments easier, and provide data that is richer for analysis, with no loss in accuracy. 
