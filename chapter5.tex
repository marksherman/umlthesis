\chapter{Discussion}
	
%\section{Conclusions}


%\section{Recommendations} 
% \label{sec:recommendations}
% We do not recommend using git as a database. Further recommendaitons of legitimate scientific and educational merit will be added as part of the full dissertation.


\section{Future Work, beyond this dissertation} % TODO reduce to just "future work" after proposal
\label{sec:futurework}

As previously mentioned, there is ample future work to be done in understanding the reasoning behind behavior choices, with particular support to misattribution of failure and success. Investigation into this misattribution should be conducted, to use traditional methods of assessing student mindset combined with the power of fine-grain snapshot analysis. 

\subsection{Better fine-grain collection of snapshot data} 
In this work, the entire project was captured on every change, which created a need to reverse-engineer what changes occured between snapshots post-facto. Upcoming updates to Blockly will allow for easier access to succinct event descriptions, potentially allowing only the changes themselves to be sent, effectively reducing the transmission payload and increasing transparency of collected data. Such improvements can make future collection experiments easier, and provide data that is richer for analysis, with no loss in accuracy. 

\subsection{Novice and Expert Behaviors}
There is also work to be done in assessing expert and novice usage patterns using snapshot analysis. \cite{petre-1995} outlined a strong difference in reading and authoring skills pertaining to secondary notation between expert and novice users. Teaching App Inventor, like any language, includes showing patterns that are good standards of practice, in hopes of the student developing more expert skills. Those patterns are not yet based on scientific observation of expert and novice programming behaviors. Snapshot analysis may be critical to provide insight into expert patterns, which could then be disseminated to enhance teaching practice. 

