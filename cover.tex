% Sherman 1
% Revision 1.1  92/04/22  13:08:20  epeisach

% BE SURE TO READ THE UNIVERSITY'S RULES ON WHAT FIELDS ARE REQUIRED OR ENCOURANGED FOR YOUR DEPARTMENT

\title{Learning Analytics for Block-based Programming: \\ The App Inventor ``Flailing Detector''}

\author{Mark A. Sherman}

\prevdegrees{M.S., University of Massachusets Lowell (2010) \\ B.S., University of Massachusetts Lowell (2008)}

\department{Department of Computer Science}

% If the thesis is for two degrees simultaneously, list them both
% separated by \and like this:
% \degree{Doctor of Philosophy \and Master of Science}
\degree{Doctor of Philosophy}
\degreemonth{March}
\degreeyear{2017}
\thesisdate{March 29, 2017}

% If there is more than one supervisor, use the \supervisor command
% once for each.
\supervisor{Fred G. Martin}{Professor}
\reader{John McCarthy}{Associate Professor}
\reader{Franklyn A. Turbak}{Associate Professor, Wellesley College}

% this is the department committee chairman, not the thesis committee chairman. 
\chairman{Haim Levkowitz}{Department Chairman}

% Make the titlepage based on the above information.  If you need
% something special and can't use the standard form, you can specify
% the exact text of the titlepage yourself.  Put it in a titlepage
% environment and leave blank lines where you want vertical space.
% The spaces will be adjusted to fill the entire page.  The dotted
% lines for the signatures are made with the \signature command.
\maketitle

% The abstractpage environment sets up everything on the page except
% the text itself.  The title and other header material are put at the
% top of the page, and the supervisors are listed at the bottom.  A
% new page is begun both before and after.  Of course, an abstract may
% be more than one page itself.  If you need more control over the
% format of the page, you can use the abstract environment, which puts
% the word "Abstract" at the beginning and single spaces its text.

%% You can either \input (*not* \include) your abstract file, or you can put
%% the text of the abstract directly between the \begin{abstractpage} and
%% \end{abstractpage} commands.

% First copy: start a new page, and save the page number.
\newpage
\thispagestyle{empty}
\mbox{}
\newpage
\thispagestyle{empty}

% Second copy: start a new page, and reset the page number.  This way,
% the second copy of the abstract is not counted as separate pages.
\pagestyle{plain}
\newpage
\setcounter{page}{1}
\pagenumbering{roman}
\begin{abstractpage}


Investigation into students' programming processes has long been a critical area to inform teaching methodology and learning models. Recent technology has created a nascent field of using fine-grained instrumentation---observations of small changes in work over time---to delve into previously inaccessible student behaviors. These behavior data filled in stories that led up to the final program code artifacts; these stories could not be captured by looking at these artifacts alone. 

Also recently, blocks-based programming languages have become popular and practical for pedagogical programming environments. These environments afford new dimensions of learner interaction with code. 

In this work, student progress during a programming challenge activity was investigated. Fine-grained data were collected based on edit operations in students' blocks-programming work. Analysis methods and visualizations for those data were designed and implemented. A construct of a class of behavior called ``flailing'' was developed, and was operationalized to include patterns of repetitive or non-productive changes. This suggested disengagement and potential lack of problem understanding. A simple algorithm was developed to demonstrate that ``flailing'' can be automatically assessed.

In addition to visualizations of individual student progress, a prototype of a classroom ``dashboard'' was developed. This is intended to be used as real-time tool for teachers, to help teachers to assess their students engagement, progress, and degree of flailing during a live lab session. This tool may be helpful to improve teaching effectiveness by aiding in classroom orchestration, empowering the teacher to better employ their resources in the moment, to better keep all students on track.

From these data, three patterns of student were discovered: those who smoothly completed the assignment, those who worked their way through with periods of flailing and periods of success, and those who only flailed and never succeeded. This rating system allowed for the development of a moment-by-moment analysis of flailing, which may provide a real-time assessment of students engagement with the activity, and serve as a tool to further empower the teacher to orchestrate their resources during a lab session.

\end{abstractpage}

%%%%%%%%%%%%%%%%%%%%%%%%%%%%%%%%%%%%%%%%%%%%%%%%%%%%%%%%%%%%%%%%%%%%%%
% -*-latex-*-



